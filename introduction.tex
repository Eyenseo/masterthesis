\documentclass[thesis]{subfiles}
\begin{document}

\chapter{Introduction}
This thesis is about the design and implementation of a \gls{skill} generator (see \autoref{sec:skill}) that targets Rust.
Through the guarantees of Rust (see \autoref{sec:rust}) the produced code will be guaranteed safe and parallelised in data-race free manner.

\section{SKilL}
  \label{sec:skill}

  \gls{skill}, as the name suggests, is a language that is intended to be used for serialisation.
  The next sections will give a brief overview of \gls{skill}, the tooling and its intended use.
  For more in depth information and the reasoning for design decisions it is advised to refer to \autocites{SKilL-TR}{SKilL-Dis}.

  \subsection{\glsentrytext{skill} Motive}
    The motive behind \gls{skill} is to provide a serialisation mothod that supports the following features\autocite{SKilL-TR}:
    \begin{itemize}
      \item small serialised file sizes
      \item fully reflective type encoding
      \item type-safe storage of references
      \item a type system that provides container, references, single inheritance and extension points
      \item tools work with partial specifications
      \item coding that is platform and language independent, upwards and downwards compatible
      \item data that is not used will not be deserialized
    \end{itemize}

    \gls{skill} itself is a data structure description language.
    With \gls{skill} it is possible to describe the data structures for a software system in a natural way.
    This is possible since the type system provides the fundamental types that are building blocks of most, commonly used languages.
    The type system in combination with the different targets (eg. C, C++, Java) allow the user to write code in the prefered or required language.

  \subsection{\glsentrytext{skill} Usage}
    After the description file is create it can be passed to the ``compiler''.
    Contrary to gcc, clang or javac the \gls{skill} compiler targets other languages like C, C++ or Java.
    To generate the target's code the first step is to translate the input to an \gls{ir}.
    That \gls{ir} is then given to a generator for further processing.
    These generators are the final step that generate the native code from the \gls{ir}.

    Generators can support different amounts of features and some features may not be native to the language eg. inheritance in C.
    In such cases the generator has to produce something compareable to the intended feature.
    In case of inheritance for C, are structures declared with all fields that were specified in the super types of them in the \gls{skill} file.
    Similarly is the upwards and downwards compatibility achived by generators.
    The basic idea is to create some kind of type erasure.
    How that is achived depends on the target language.

\section{Rust}
  \label{sec:rust}
  % TODO mention
  % Like C++, Rust takes zero-cost abstractions as one of its core principles:
  % none of Rust’string abstractions impose a global performance penalty, nor is
  % there overhead from any runtime system in the traditional sense.

  The official Rust language website describes the language as;
  \enquote{Rust is a systems programming language that runs blazingly fast, prevents segfaults, and guarantees thread safety.}\autocite{rustorg}
  It presents itself mostly as a \emph{safe} language that should be considered as substitute for C, C++ and other systems programming languages.
  The project criticizes especially the following points in other languages and intends to solve them:
  \begin{enumerate}
    \item \enquote{There is too little attention paid to safety.}\autocite{rustorg}
    \item \enquote{They have poor concurrency support.}\autocite{rustorg}
    \item \enquote{There is a lack of practical affordances.}\autocite{rustorg}
    \item \enquote{They offer limited control over resources.}\autocite{rustorg}
  \end{enumerate}
  But the project doesn't aim to create a complete safe language and intends to make compromises in favor of other goals.
  % TODO mention Ada here?

  The following sections will provide a short introduction to the Rust programming language.
  The focus is on the elements and features that are important to this thesis and with that non exhaustive.
  For a more in-depth and up to data introduction it is strongly advised to refer to the official documentation.

  \subsection{Borrows and Lifetimes}
    At the core of the memory and concurrency goal stands the \emph{borrowing} mechanic.
    Borrowing can be thought of as \emph{taking a reference} in C++.
    Like all variables borrows are by default \emph{immutable} or in C++ words \codc{const}.
    Contrary to C++ mutable access has to be explicitly requested with the \codr{mut} keyword.
    Additionally to the reference Rust also attaches a \emph{lifetime} to the borrow.
    This lifetime can be thought of as a `reference counted`\footnote{%
      This will be a more fitting mental image once non-lexical lifetimes become stable.
    } smart-pointer \emph{at compile-time}.
    A program will fail to compile if there is more than one mutable access to a variable or a writable and readable access exists at the same time.

    \lstinputlisting[
      style=myEnvStyle,
      language=myRust,
      float,
      floatplacement=ht,
      caption={Rust Borrow Mechanics},
      label={lst:borrowing},
    ]{data/borrow_mut.rs}

    \autoref{lst:borrowing} depicts a simple example that demonstrates these rules.
    In line 2 a new \emph{owned} string is created -- \codr{"Hello"} is a string literal and can't be modified.
    Since the string shall be modified an explicit conversion function has to be called.
    An immutable borrow is taken from \codr{string} in line 2.
    It is fine to use both \codr{string} and \codr{foo} to call methods on the object that require an immutable borrow.
    But using a method that requires a mutable borrow on \codr{foo} will lead to a compile-time error.
    It might be surprising, but line 9 also leads to a compile-time error\footnote{%
      With non-lexical lifetimes this will compile.
    }.
    Since \codr{string} is borrowed immutable until the end of the scope it doesn't matter that it is a mutable variable.

    \newsavebox{\lifetimeErrBox}
    \begin{lrbox}{\lifetimeErrBox}% Store first listing
      \begin{minipage}{.45\linewidth}
        \lstinputlisting[
          style=myEnvStyle,
          language=myRust,
          nolol=ture,
        ]{data/lifetime_err.rs}
      \end{minipage}
    \end{lrbox}

    \newsavebox{\lifetimeFixBox}
    \begin{lrbox}{\lifetimeFixBox}% Store first listing
      \begin{minipage}{.45\linewidth}
        \lstinputlisting[
          style=myEnvStyle,
          language=myRust,
          nolol=ture,
        ]{data/lifetime_fixed.rs}
      \end{minipage}
    \end{lrbox}

    \begin{figure}[ht]
      \captionsetup{type=lstlisting}
      \subfloat[Explicit Lifetime Required]{\usebox{\lifetimeErrBox}\label{lst:lifetimeErr}} \hfill%
      \subfloat[Lifetime Mechanic Benefits]{\usebox{\lifetimeFixBox}\label{lst:lifetimeFix}}
      \caption{Rust Lifetime Mechanics}\label{lst:lifetime}
    \end{figure}

    In \autoref{lst:borrowing} the lifetimes can be omitted as the compiler is able to infer them.
    For \autoref{lst:lifetimeErr} an explicit lifetime is needed for the attribute \codr{s} in the struct \codr{Foo}.
    Here the compiler is not able to infer where the lifetime for \codr{s} will come from.
    In \autoref{lst:lifetimeFix} this error is fixed by explicitly giving the borrow a lifetime.
    % TODO is it really or has it a special name?
    This leads to a different compiler error that is called \emph{use after free} and can lead to dangerous situations.
    At this point it has to be noted thou that lifetimes can only be applied to borrows.
    In other words they can't be used to say something about the lifetime of memory itself.

  \subsection{Structs, Traits and Generics}
    The fundamental pieces of the Rust \gls{oop} model are \codr{struct}s, \codr{trait}s and \codr{impl} blocks.
    \codr{struct}s are equivalent to C \codc{struct}s as they are a collection of fields.
    \codr{trait}s can be thought of as Java \codj{interface}s, they dictate a set of functions that have to be implemented by a type that provides a certain \codr{trait}.
    Additionally to \codr{trait}s \codr{impl} blocks can be used to add methods to a \codr{struct}.

    While Rust can be thought of as an \gls{oop} language it differs in a few features these languages exhibit usually.
    The most apparent is the inheritance model.
    While it is not possible to inherit from a \codr{struct}, \codr{trait}s can inherit each other.
    It is also possible to implement functions and \codr{trait}s for a set of types through generic type parameter.
    This set of types is made up of a types that satisfy a collection of \codr{trait}s and lifetimes.
    These are also called \emph{bounds} and one can be seen in line 5 of \autoref{lst:genericsCall}.
    Never the less leads the missing inheritance of \codr{struct}s to `forced' composition and to code duplication if a bound can't be expressed.
    Such a situation can appear as a bound has to hold for all types, whether all types actually invoke the functions doesn't matter.
    This is in contrast to C++ where only the function invocation triggers a check for it.
    In cases like this there are macros that provide a convenient way to duplicate the code at the appropriate places.
    % TODO mention 'missing' function overloading

    \newsavebox{\genericsBase}
    \begin{lrbox}{\genericsBase}%
      \begin{minipage}{.45\linewidth}
        \lstinputlisting[
          style=myEnvStyle,
          language=myRust,
          nolol=ture,
          lastline=15,
        ]{data/generics.rs}
      \end{minipage}
    \end{lrbox}

    \newsavebox{\genericsCall}
    \begin{lrbox}{\genericsCall}
      \begin{minipage}{.45\linewidth}
        \lstinputlisting[
          style=myEnvStyle,
          language=myRust,
          nolol=ture,
          firstline=17,
          lastline=31,
        ]{data/generics.rs}
      \end{minipage}
    \end{lrbox}

    \begin{figure}[ht]
      \captionsetup{type=lstlisting}
      \subfloat[Type Structure]{\usebox{\genericsBase}\label{lst:genericsBase}} \hfill%
      \subfloat[Generics and Trait Usages]{\usebox{\genericsCall}\label{lst:genericsCall}}
      \caption{Rust Generics and Trait Objects}\label{lst:generics}
    \end{figure}

    \autoref{lst:genericsBase} declares a \codr{trait} and two \codr{struct}s that then implement the \codr{trait} respectively.
    In line 1 of \autoref{lst:genericsCall} the function \codr{fun_gen} uses a type parameter \codr{T} with a bound of \codr{Who}.
    There too happens code duplication -- this time on invocation of the generic function.
    This is called \emph{monomorphism} and behaves similarly the way C++ realises template functions.
    In case that a function is not implemented for a bound, or it is not possible to do so, the alternative is a \emph{trait object}.
    This can be seen in line 5 of \autoref{lst:genericsCall} where the function expects an trait object.
    Trait objects can be compared to abstract classes in C++.
    This means that method calls are a \emph{dynamic dispatched} -- in other words a virtual function call is executed for \codr{obj.who()}.
    Since virtual function calls are generally slow it often advised to write code that is based on bounds.
    Rusts approach to virtual function calls is based on fat pointers \cod{[data_ptr, vtable_ptr]}.
    This might introduce additional slowdown as more cache is need per object.

  \subsection{Errors and Null}
    For errors, failures or exceptions Rust doesn't use \emph{exceptions}\footnote{%
      This is partially true as \codr{panic}s can be caught\ldots
    }.
    Rust provides two ways to express an error, both can be seen in \autoref{lst:errorFun}.
    The first way uses the \codr{Result<T, E>} wrapper as return type of functions.
    This type contains either the success return value of type \codr{T} or an error value of type \codr{E}.
    In case the value is \codr{0} the success value will be returned.
    In case that the value is greater than \codr{7} an error value will be returned instead.
    Lastly there is \codc{panic!}\footnote{%
      The \codr{!} indicates that \codr{panic!} is a macro.
    } which will terminate the program.
    This macro is used throughout the standard library e.g.\ for an access over the limits of an array.
    Another wrapper type that is used throughout the standard library is \codr{Option<T>}.
    The main difference to \codr{Result} is, that \codr{Option} can only hold a valid value or non at all.
    An wrapped value is created thought \codr{Some(val)} or \codr{None}.

    \newsavebox{\errorFun}
    \begin{lrbox}{\errorFun}%
      \begin{minipage}{.45\linewidth}
        \lstinputlisting[
          style=myEnvStyle,
          language=myRust,
          nolol=ture,
          lastline=9,
        ]{data/error.rs}
      \end{minipage}
    \end{lrbox}

    \newsavebox{\errorGun}
    \begin{lrbox}{\errorGun}
      \begin{minipage}{.45\linewidth}
        \lstinputlisting[
          style=myEnvStyle,
          language=myRust,
          nolol=ture,
          firstline=10,
        ]{data/error.rs}
      \end{minipage}
    \end{lrbox}

    \begin{figure}[ht]
      \captionsetup{type=lstlisting}
      \subfloat[Rust Error Types]{\usebox{\errorFun}\label{lst:errorFun}} \hfill%
      \subfloat[Ways to Deal with Errors]{\usebox{\errorGun}\label{lst:errorGun}}
      \caption{Rust Error Mechanics}\label{lst:error}
    \end{figure}

    \autoref{lst:errorGun} shows two ways how these wrapped values can be unwrapped though \emph{matching}.
    In \codr{gun()} it is the \codr{?} operator, which unwraps the \codr{Result}.
    If it is the success value it will be returned to the \emph{scope}.
    But if it is the error value the value will be returned from the \emph{function}.
    The \codr{?} operator is a tool to direct the data flow of possible error results conveniently.
    What the \codr{?} operator does can be roughly seen in the \codr{hun()} function.
    In case of an \codr{Option} there is a more convenient way; \codr|if let Some(val) = val { ... }|.
    Matching is can be found in many aspects of the language and is is mostly a convenience feature.

  \subsection{Tools and Ecosystem}
    Another place where Rust differentiates itself from competitors is the tooling and the resulting ecosystem.
    As a young language Rust took the opportunity to learn and improve upon the established languages.

    The build system is not a third party tool but rather an extension of Rust language itself.
    Projects are created through the \cod{cargo} command that creates a bare bones project structure and also a \cod{git} repository.
    Through \cod{cargo} builds, tests, benchmarks and publications are handled.
    The tool handles the dependencies and downloads the specific libraries from \href{www.crates.io}{crates.io} which is something that other systems languages struggle with.
    Rust also paid close attention to documentation and tests code inside documentation e.g.\ code examples.
    Further is the ability to generate webpages with the documentation build-in which is similar to Python.

    Through the appealing goals and the convenient tooling Rust gained traction over the last years.\autocites{sosurvey}{rustorg}{rust3}
    It also sees application in the industry slowly.
    Mozilla's Firefox uses Servo, the \gls{css} Style engine, since the 57 Firefox release.\autocite{servo}
    Other notable companies that are using Rust are Atlassian, Cloudflare
    and Dropbox.\autocite{rustorg}
    While the industry is slowly testing and adopting Rust it has to be said that this language is still young and undergoes radical changes ever so often.

\end{document}
