\documentclass[thesis]{subfiles}

\begin{document}

\chapter{Introduction}\label{sec:intro}
This thesis is about the design and implementation of a \gls{skill} generator that targets Rust.
\gls{skill} is a collection of specifications and tools that enables users to write their applications in their preferred language while offering the ability to serialize the applications state platform and language independently\autocite{skill-tr}.
Language independence is achieved by offering multiple language targets for bindings that are generated from a user provided specification.
Rust was chosen as it aims to be a new, safer, systems programming language that features a type system that offers thread safety guarantees\autocite{rust-org}.
These should allow for safe parallelism in the generated bindings.
As reference implementation C++ was used, but C, Java and Scala were available too.
This section will introduce \gls{skill} and Rust.
In the following section that the design and implementation will be presented, as well as issues and solutions to them.
Finally the results will be evaluated through functional and performance tests.

\section{SKilL}\label{sec:skill}
  \gls{skill} encompasses a specification, binary format and tooling for them, that are used to generate code which enables users to serialise data of their software.
  For this thesis serialization is used for both reading data from files and writing data to files.
  The next sections will give a brief overview of features, tooling and intended use of \gls{skill}.
  For more in depth information and the reasoning for design decisions it is advised to refer to \citetitle{skill-tr}\autocite{skill-tr} and \citetitle{skill-dis}\autocite{skill-dis}.

  \subsection{Motivation}
    \gls{skill} has been created with the goal to provide small serialised file sizes, fully reflective type encoding, type-safe storage of references, a rich type system, tools that work with partial specifications,
    coding that is platform and language independent, upwards and downwards compatibility and lazy reading of unused data.
    The specification language was kept simple by providing fundamental types that are building blocks of most, commonly used languages as well as single inheritance.
    These properties allow users to describe the data structures for a software system in a natural way.
    This specification is then used to generate a binding in a language chosen by the users e.g.~C, C++, Java.
    With the generated binding users are able to write their implementation in a language that they are most comfortable with, by using the offered \gls{api}.
    \autocite{skill-tr}

  \subsection{Architecture}\label{sec:skillArch}
    \autoref{fig:archDesire} illustrates the fundamental architecture of a \gls{skill} binding.
    The \SkillFile is the \gls{api} root that the user accesses to crate and open \gls{skill} files.
    Through \SkillFile instances the user can access a management object, from here on called \StringPool, that is responsible to create and manage \Strings in such a way that they can be serialised.
    Likewise there are management objects that are responsible for instances of a user defined type, from here on called \UserTypePools.
    These management objects are necessary as the runtime system has to know which objects are to be serialized.
    For this reason deletion of instances is also handled through these management objects.
    \autocite[107--112]{skill-dis}

    \begin{figure}[ht]
      \centering
      \begin{tikzpicture}[scale=0.775]
  \begin{scope}[
      every node/.style = {
          shape=rectangle, rounded corners, draw,
          align=center, font=\ttfamily, minimum height=1.825em,
        },
      edge from parent/.style = {
          draw=none,
        },
      level distance=1.75cm,
      sibling distance=1cm,
    ]

    \Tree[
    .\node(SkillFile){SkillFile};
    [
    .\node(StringPool){StringPool};
    [
    .\node(String){String};
    ]
    ][
    .\node(UserTypePool){UserTypePool};
    [
    .\node(UserType){UserType};
    ][
    .\node(FieldDeclaration){FieldDeclaration};
    ]
    ][
    .\node(FileReader){FileReader};
    [
    .\node(FileWriter){FileWriter};
    ]
    ]
    ]
  \end{scope}

  \begin{scope}[
    >={Stealth[black]},
    every node/.style={font=\ttfamily,scale=0.775},
    every edge/.style={line width=.3mm},
    create/.style={->[Coral],draw,Coral,dashed,>={Latex}},
    has/.style={->[SteelBlue],draw,SteelBlue,>={Latex}},
    ]
    % NOTE has
    \path (SkillFile) edge[has,bend left=20] (FileReader);
    \path (SkillFile) edge[has,bend left=10] (FileWriter);


    \path (SkillFile) edge[has,bend left=20] (UserTypePool);
    \path (UserTypePool) edge[has,bend right=20] (UserType);
    \path (UserTypePool) edge[has, out=290, in=140] (FieldDeclaration);

    \path (FieldDeclaration) edge[has,out=110,in=320] (UserTypePool);
    \path (FieldDeclaration) edge[has,bend right=7] (StringPool);

    \path (SkillFile) edge[has,bend right=20] (StringPool);
    \path (StringPool) edge[has] (String);

    \path (UserType) edge[has] (String);
    \path (UserType) edge[has,loop below, distance=.75cm] (UserType);

    \path (FieldDeclaration) edge[has,bend left=10] (FileReader);

    % NOTE create
    \path (FileReader) edge[create] (UserTypePool);
    \path (FileReader) edge[create,bend right=10] (StringPool);
    \path (FileReader) edge[create,bend left=10] (FieldDeclaration);

    % Legend
    \coordinate[above left=.25 and 3.5of SkillFile](Legend1A);
    \coordinate[above left=.25 and 4.25of SkillFile](Legend1B);
    \coordinate[below=.5of Legend1A](Legend2A);
    \coordinate[below=.5of Legend1B](Legend2B);
    \path (Legend1A) edge[create] (Legend1B) node[anchor=west,create,draw=none]{create};
    \path (Legend2A) edge[has] (Legend2B) node[anchor=west,has,draw=none]{has};
  \end{scope}
\end{tikzpicture}

      \caption{Abstract \gls{skill} Binding Architecture}\label{fig:archDesire}
    \end{figure}

    \UserTypePools contain \FieldDeclarations that describe fields of \UserTypes.
    They can't be encoded in the \UserTypePools at generation time, as there can be fields that have not been known at that time.
    Further have fields to be validated against the ones in the specification.
    For serialization \FieldDeclarations have to know the \UserTypePools as \UserType fields can have \UserTypes as fields.
    Similarly do they have to know the \StringPool in case that a field is of type \String.
    \autocite[144--148]{skill-dis}

    The \FileReader is responsible for the creation of the \StringPool, \UserTypePools and \FieldDeclarations.
    While the creation of \UserType instances is handled by \UserTypePools, before field data is read, \FieldDeclarations are responsible for the reading field data.
    This split is done as only \FieldDeclarations can know how to read the data and if the reading can be deferred until the first access or writing of the instance.
    Furthermore allows this two step process potentially for parallelization as all \UserType instances are guaranteed to exist when \FieldDeclarations require them.
    \autocite[148f.]{skill-dis}

  \subsection{Serialization}\label{sec:rwSkill}
    \begin{wrapfigure}{R}{0.4\textwidth}
      \centering
      \vspace*{-\baselineskip}
      \begin{tikzpicture}
        \tikzset{
          node distance=1mm,
          box/.style={draw, rounded corners=.5em},
          block/.style={box, minimum width={width("Field Meta Data")+3mm}},
          string/.style={box, minimum width={width("Lengths")+3mm}},
          type/.style={box, minimum width={width("Field Meta Data")+3mm}},
        }
        \node(name) {Skill File};
        \node[box, below=-.5em of name] {
          \begin{tikzpicture}
            \node(name) {Block$_i$};
            \node[box, below=of name] {
              \begin{tikzpicture}
                \node[block] (stringBlock) {
                  \begin{tikzpicture}
                    \node(name) {String Block};
                    \node[string, below=of name] (lengths) {Lengths};
                    \node[string, below=of lengths] {Strings};
                  \end{tikzpicture}
                };
                \node[block, below=of stringBlock] (typeBlock) {
                  \begin{tikzpicture}
                    \node(name) {Type Block};
                    \node[type, below=of name] (types) {Type Meta Data};
                    \node[type, below=of types] (meta) {Field Meta Data};
                    \node[type, below=of meta] {Field Data};
                  \end{tikzpicture}
                };
              \end{tikzpicture}
            };
          \end{tikzpicture}
        };
      \end{tikzpicture}
      \captionsetup{justification=raggedright}
      \caption{\gls{skill} File Layout}
      \label{img:skilllayout}
      \vspace*{-\baselineskip}
    \end{wrapfigure}
    As seen in \autoref{img:skilllayout} a \gls{skill} file can consist of multiple blocks.
    Each of these blocks consists of a string section and a type section.
    String sections first list the amount of strings, followed by the lengths of the strings, followed by the strings themself.
    This layout allows for easy reading and writing of the strings with \cod{mmap}\footnote{
      \cod{mmap} allows to map files and file sections to a memory location that can be accessed like an array.
    } and potentially for parallelization.
    Type sections start with the amount of types followed by the type definitions.
    These definitions are about the types only and do not contain information about the fields of the types.
    In that section the type name, number of instances, restrictions, number of fields and super type are listed.
    The next section contains the field definitions for the types.
    Similar to inheritance in most other languages, super field definitions are not redefined for the children types.
    The definition consists of the field name, type of the field, as well as the restrictions.
    A field definition ends with an offset into the next section where the data of the fields resides.
    \autocite{skill-tr}

    All data touched so far is small and fast to read as its size is not influenced by the amount of instances.
    It is mostly relevant to the \gls{skill} runtime system as this information allows it to serialise data.
    The last section contains field data of serialised type instances where data of one field is written contiguously for all instances that have that field.
    The known offsets from the previous sections field declarations and the specific layout of the field data can then be used to skip data with \cod{mmap} to allow for parallelized serialization.
    This is possible as there is no overlapping write and read access on the instances memory as all instances are created before serialization starts.
    \autocite{skill-tr}

    The reading and writing  operations themselves are not too different from the other implementations.
    It will not be explained in detail.
    Reading of multi-block files is more sophisticated than this section may make them out to be, but it is not crucial for the rest of the thesis.
    The involved data structures and data flow however will be discussed as well as their consequences.

  \subsection{Generators}
    The generators are written in Scala and rely on mixins to be easily extensible, and string interpolation for readability.
    Mixin classes are a kind of inheritance that allows linearisation of subclasses with different super types instead of an inheritance tree\autocite[117f.]{oop-inter}.
    Each generator has a \cods{Main} \cods{class} that has various \cods{trait}s mixed in.
    These traits are then responsible to create a certain set of files.

    Generators work with an \gls{ir} that provides all types that have been defined by the user as well as types for the build-in types.
    These type representations are given to the generator in form of a type context which can be queried for different types, e.g.~user types or interfaces.
    Once the needed lists of type representations is obtained a generator can create the files with the translated type definitions.
    Generators have to pay special attention to the escaping of reserved keywords in their target language.
    For that a escaping function can be used that is applied to non literal strings.

\section{Rust}\label{sec:rust}
  The official Rust language website describes the language as:
  \enquote{Rust is a systems programming language that runs blazingly fast, prevents segfaults, and guarantees thread safety.}\autocite{rust-org}
  It presents itself as a \emph{safe} language that should be considered as substitute for C, C++ and other systems programming languages.
  The project criticizes especially the following issues in other languages and intends to solve them\autocite{rust-faq}:
  \begin{enumerate}
    \item There is too little attention paid to safety.
    \item They have poor concurrency support.
    \item There is a lack of practical affordances.
    \item They offer limited control over resources.
  \end{enumerate}
  But the project doesn't aim to create a completely safe language and instead intends to make compromises in favor of other goals.
  \autocite{rust-faq}

  The following sections will provide a short introduction to the Rust programming language.
  The focus is on the elements and features that are important to this thesis.
  For a more in-depth and up to date introduction it is strongly advised to refer to the official documentation\autocites{rust-org}{rust-doc}{rust-nom}{rust-book}.

  \subsection{Borrows and Lifetimes}\label{sec:rustBorr}
    At the core of the memory and concurrency goal stands the \emph{borrowing} mechanic.
    Borrowing can be thought of as \emph{taking a reference} in C++.
    Like all variables borrows are by default \emph{immutable} or in C++ words \codc{const}.
    Contrary to C++ mutable access has to be explicitly requested with the \codr{mut} keyword.
    Where borrows differ from references is that they behave like a reader-writer lock regarding the mutability state of the referenced value.
    That means that a mutable variable may be referenced immutable more than once at a time or exclusively mutable.
    This is referred to as \emph{borrow rules}.
    Additionally Rust attaches a \emph{lifetime} to the borrow that is the lifetime of the referenced memory allocation.
    A borrow may not outlive its lifetime, meaning that there can't be a reference to freed memory.

    \LstTikzFig{data/borrow_mut.rs}{Rust Borrow Mechanics}{lst:borrowing}

    \autoref{lst:borrowing} depicts a simple example demonstrating these rules.
    In line 2 a new \emph{owned} string is created -- \codr{"Hello"} is a string literal and can't be modified.
    Since the string shall be modified an explicit conversion function has to be called.
    An immutable borrow is taken from \codr{s} in line 3 with the \codr{&} operator.
    Here the lifetime can be omitted as the compiler is able to infer it.
    It is fine to use both \codr{s} and \codr{rs} to call methods on the object that require an immutable borrow.
    But, using a method that requires a mutable borrow on \codr{rs} will lead to a compile-time error.
    It might be surprising, but line 8 also leads to a compile-time error.
    Since \codr{s} is borrowed immutable until the end of the scope it doesn't matter that it is a mutable variable.

    \LstTikzBox{\lifetimeErrBox}[.4425\linewidth]{data/lifetime_err.rs}
    \LstTikzBox{\lifetimeFixBox}[.44\linewidth]{data/lifetime_fixed.rs}
    \begin{figure}[ht]
      \captionsetup{type=lstlisting}
      \subfloat[Explicit Lifetime Required]{\usebox{\lifetimeErrBox}\label{lst:lifetimeErr}}\hfill%
      \subfloat[Lifetime Mechanic Benefits]{\usebox{\lifetimeFixBox}\label{lst:lifetimeFix}}
      \caption{Rust Lifetime Mechanics}\label{lst:lifetime}
    \end{figure}

    In \autoref{lst:lifetimeErr} a \struct \Foo is created that has an attribute.
    \codr{s} is of type \codr{&String}, or in words: \emph{\codr{s} is a reference to an immutable \String}.
    For \codr{s} an explicit lifetime is needed because the compiler is not able to infer where the lifetime will come from.
    In \autoref{lst:lifetimeFix} this error is fixed by explicitly giving the borrow a lifetime with \codr{'a}.
    The lifetime is given to the attribute through the type signature of \Foo.
    On construction of \Foo, line 7, the lifetime of \codr{s} is used as the lifetime for the reference.
    Because \codr{s} is freed at the end of the scope in line 9 and the \Foo instance is not, a compilation error is raised.
    Since the memory of \codr{s} was freed an access through \codr{foo.s} could lead to runtime failures.
    It has to be noted though that lifetimes can only be applied to borrows and are taken implicitly.
    In other words they can't be used to chain two instances to each other without using borrows\autocite[PhantomData]{rust-doc}.

  \subsection{Structs, Traits and Generics}\label{sec:stg}
    The fundamental pieces of Rusts \gls{oop} model are \structs, \traits and \codr{impl} blocks.
    \structs are equivalent to C \structs as they are a collection of fields.
    \traits can be thought of as Javas \codj{interface}s.
    They define a set of functions that have to be implemented by a type that provides a certain \trait.
    Besides \traits, \codr{impl} blocks can be used to add methods to \structs.

    While Rust can be thought of as an \gls{oop} language it differs in a few features these languages normally offer.
    Inheritance is one of the features that differ.
    It is not possible to inherit from a \struct, \traits however can inherit each other.
    \traits can also be implemented for a set of types through type parameters.
    The set of types that a trait is generically implemented for can be restricted through a collection of required \traits and lifetimes.
    These requirements are also called \emph{bounds} and can be seen in line 16 of \autoref{lst:genericsCall}.
    As result of the missing inheritance code reuse can be more difficult if a bound can't be expressed.
    Such a situation can appear as a bound has to hold for all types, whether all types actually invoke the function doesn't matter.
    This is in contrast to C++ where only the function invocation triggers the instantiation for it\autocite[§17.8.1 10]{cpp-iso} which allows more code to be valid.
    As solution either composition or code duplication can be used, where the later can be done through macros.
    \autocite[10. Generic Types, Traits, and Lifetimes, 17. Object Oriented Programming Features of Rust]{rust-book}

    \LstTikzBox{\genericsBase}[.42\linewidth]{data/generics.rs}[lastline=14]
    \LstTikzBox{\genericsCall}[.42\linewidth]{data/generics.rs}[firstline=16,firstnumber=16]
    \begin{figure}[ht]
      \captionsetup{type=lstlisting}
      \subfloat[Type Structure]{\usebox{\genericsBase}\label{lst:genericsBase}} \hfill%
      \subfloat[Generics and Trait Usages]{\usebox{\genericsCall}\label{lst:genericsCall}}
      \caption{Rust Generics and Trait Objects}\label{lst:generics}
    \end{figure}

    \autoref{lst:genericsBase} declares a \trait and two \structs that then implement the \trait respectively.
    The functions \codr{who} takes a parameter \codr{&self} which de-sugars to \codr{self: &Pinky} and \codr{self: &Becci}.
    In line 1 of \autoref{lst:genericsCall} the function \codr{fun_gen} uses a type parameter \T with a bound of \codr{Who}.
    There happens code duplication too -- this time on invocation of the generic function\footnote{
      No official documentation validates this, but the assembly code does. With optimisations both functions are in-lined.
    }.
    \T will be substituted by the \struct that implements the bound -- in this case \codr{Pinky} and \codr{Becci}.

    In case that a function is not implemented for a set of types, or it is not possible to do so, the alternative is a \emph{trait object}\autocite[17. Object Oriented Programming Features of Rust]{rust-book}.
    This can be seen in line 19 of \autoref{lst:genericsCall} where the function expects a trait object of type \codr{Who}.
    Trait objects can be compared to virtual abstract classes in C++.
    This means that method calls can be \emph{dynamic dispatched} -- in other words a virtual function call can be executed for \codr{obj.who()}.
    Rusts approach to virtual function calls is based on fat pointers \cod{[data_ptr, vtable_ptr]}\autocites[std::raw::TraitObject]{rust-doc}[Trait Objects]{rust-book1}.
    This can introduce additional slowdown as more cache is needed per object, access is non atomic\autocite[8.2.3.1 ff.]{intel64and} and moves are more expensive.

  \subsection{Errors, Null and Matching}\label{sec:rustErr}
    Rust provides two ways to express an error, both can be seen in \autoref{lst:errorFun}.
    The first way uses the \codr{Result<T, E>} wrapper as return type of functions.
    This type contains either the success return value of type \T or an error value of type \codr{E}.
    In case the value is \codr{0} the success value will be returned.
    In case that the value is greater than \codr{7} an error value will be returned instead.
    This value can be a \struct and contain detailed information or non at all.
    Lastly, there is \codc{panic!}\footnote{%
      The \codr{!} indicates that \codr{panic!} is a macro.
    } which will terminate the program.
    A panic in Rust can be compared to an exception, as a panic can be caught\autocite[std::panic::catch\_unwind]{rust-doc}.
    This macro is used throughout the standard library e.g.~for an access over the limits of an array.
    But panics are mostly used for `unrecoverable' or unexpected problems, whereas \codr{Result} is used in case an error can be expected, e.g.~\codr{File::open()} returns \codr{Result<File, Error>}.
    Another wrapper type that is used throughout the standard library is \OptionT, seen in the \codr{gun} function.
    The main difference to \codr{Result} is, that \codr{Option} can only hold a valid value or non at all.
    Because of that it is also used as replacement for \codc{NULL}.

    \LstTikzBox{\errorFun}[.45\linewidth]{data/error.rs}[lastline=16]
    \LstTikzBox{\errorGun}[.4\linewidth]{data/error.rs}[firstline=17,firstnumber=17]

    \begin{figure}[ht]
      \captionsetup{type=lstlisting}
      \subfloat[Rust Error Types]{\usebox{\errorFun}\label{lst:errorFun}} \hfill%
      \subfloat[Ways to Deal with Errors]{\usebox{\errorGun}\label{lst:errorGun}}
      \caption{Rust Error Mechanics}\label{lst:error}
    \end{figure}

    \autoref{lst:errorGun} shows three ways how these wrapped values can be unwrapped though \emph{matching}.
    In \codr{hun()} it is the \codr{?} operator, which unwraps the \codr{Result}.
    If it is the success value it will be returned to the \emph{scope}.
    But if it is the error value the value will be returned from the \emph{function}.
    The \codr{?} operator is a tool to direct the data flow of possible error results conveniently.
    What the \codr{?} operator does can be roughly seen in the \codr{nun()} function.
    In \codr{pun} the third way can be seen, where the \OptionT value is moved and if it is \Some it will be matched and accessible through \codr{didi}.
    Matching can be found in many aspects of the language and is mostly a convenience feature as functions can be used to unwrap the values as well.
    \autocite[Enums and Pattern Matching]{rust-doc}

  \subsection{Visibility and Grouping}
    Visibility in Rust is specified with the \pub keyword.
    To specify that a field is public to the crate it is prefixed as \codr{pub(crate) foo: String}.
    Apart from fields and types \pub is also used for modules.
    Modules can be compared to Javas packages or C++ namespaces.
    Files themselves are a module and further modules can be declared in files with the \mod keyword.
    This can be used to create a more fine grained grouping and visibility resolution.
    To flatten the module hierarchy, in case that the file module is not wanted, or to create a more convenient user \gls{api} the \use keyword can be used to pull types in another module.
    \autocite[Visibility and Privacy, Use declarations]{rust-ref}

  \subsection{Cells and Interior Mutability}\label{sec:cell}
    \UnsafeCellT is a wrapper type from Rusts standard library.
    This wrapper allows to change contained data without having to be mutable.
    In cases where an object contains a \UnsafeCellT there are two mutabilities, interior and exterior.
    In other words: an object that is immutable doesn't deny the modification of a value contained in a \UnsafeCellT.
    The result is, that code that uses an \UnsafeCellT is vulnerable to most issues that Rust wants to prevent.
    It is in the responsibility of the developer to ensure that the borrowing rules are heeded to, as well as thread safety is guaranteed.\autocite[std::cell]{rust-doc}

    \LstTikzFig{data/cell.rs}{Example to Interior Mutability}{lst:cell}

    In \autoref{lst:cell} an example to interior mutability can be seen.
    The \struct \Foo contains a \CellT, which uses internally a \UnsafeCellT.
    The provided interface of \CellT is safe to use and doesn't need \unsafe like \UnsafeCellT does because there is not way for the user to borrow the contained value.
    In line 7 it is possible to modify the value of \x.
    The value of \y in line 8 however raises the compiler an error.
    The mutability of \x would lead to data races and because of that \CellT and \RefCellT are not marked as thread safe\autocites[std::cell]{rust-doc}[Send and Sync]{rust-nom}.

\section{Rust Oddities}
  This section will focus on oddities and questionable design choices that have been encountered while working with Rust.
  While these are not strictly necessary for the understanding of this thesis, they do provide a deeper understanding of Rust which is helpful to follow some decisions.

  \subsection{Immutable File Manipulations}
    A few strange things happen in \autoref{lst:rustfile}.
    The first thing is, that line 8 and 10 produce \emph{not} the same output.
    \codr{file} is immutable, so why and how can an iterator, created by \codr{bytes(self)}, change the position in the file?
    The how can be explained with interior mutability which was introduced in the previous section.
    The why however is left open.

    Further how can bytes be called by \codr{&file}?
    \codr{&file} creates a reference to an immutable file but \codr{bytes(self)} requires an object.
    The thing that is deceptive is that \trait \Read is not implemented for \File but for \codr{&'a File}.
    Which means that \self is not an object but a reference to one.

    But the real question is the following: How can it be that line 12 fails?
    With the results, that prompt the first question, this should be allowed.
    In this case there is really no modification going on too.

    \LstTikzFig{data/file.rs}{File is not Subject to Borrow Rules}{lst:rustfile}

  \subsection{Immutability Resolution}
    Rusts immutability is an on off switch, there is no mechanism to define an attribute as immutable.
    So if an object is passed as immutable nothing can be changed, that is not wrapped by an \UnsafeCellT.
    But on the other hand everything can be changed if the object is mutable.
    Coming from a C++ background the lack of the \codc{const} feature reduces the expressiveness.
    To a certain degree it also influences the safety of the program by providing another level of type safety\autocite[30ff.]{cpp101}.
    A workaround is an extra type that wraps the value and only provides immutable accessors.

  \subsection{Static Functions for Traits}
    Implementing a static function for a \trait should not be difficult, but it is.
    In \autoref{lst:static} two options are shown how-to do it.

    The first option involves functions that are implemented for \Foo as well as the two \traits without a \self parameter.
    For the \traits a bound is required that specifies that the \self type (\Self) has to be \Sized, meaning that the size of \Self is known at compile time.
    Unfortunately they can't be called on the \trait itself, as seen in line 28, because of that bound.
    Valid calls for this approach can be seen in lines 31 to 33.

    The other option uses an extra \trait that is implemented for the trait that should have the static function available.
    In this example the \trait is named \codr{IDs} and the expected call syntax works as seen in lines 35 to 37.
    An alternative for static functions of \traits are top-level functions that are grouped by name or module.
    This however is just an option in a context that doesn't use type parameters where the \trait should be used.

    \LstTikzBox{\staticOne}[.47\linewidth]{data/static.rs}[lastline=12]
    \LstTikzBox{\staticTwo}[.4\linewidth]{data/static.rs}[firstline=14,firstnumber=14,lastline=25]
    \LstTikzBox{\staticThree}{data/static.rs}[firstline=27,firstnumber=27]
    \begin{figure}[ht]
      \captionsetup{type=lstlisting}
      \centering
      \usebox{\staticOne}\hfill%
      \usebox{\staticTwo}\\\vspace*{.75em}%
      \usebox{\staticThree}
      \caption{Static Functions for Traits}\label{lst:static}
    \end{figure}

  \subsection{Coercion of \texttt{trait}s}
    Rust uses \emph{coercion} as the name for implicit casts.
    \autoref{lst:coercionTwo} shows strange mechanic that is related to coercion.
    Here, \A is inherited by \B and \B and \A are implemented for \Foo.
    Then \codr{x} is borrowed as \codr{&B} and through \codr{rb} the function \fun of \A is called successfully.\footnote{
      \codr{rb} and \codr{ra} have to be borrows as their size is not known at compile time.\autocite[E0277 referes to 19.4 Advanced Types]{rust-doc}
    }
    But in line 12 the compiler raises an error, informing us that \codr{rb} is not of type \A.
    Further is it not possible to cast \codr{rb} to \codr{A&} with the \codr{as} keyword, as seen in line 13.

    \LstTikzFig{data/coercion_2.rs}{Coercion of Traits}{lst:coercionTwo}

  \subsection{Lifetimes}
    Lifetimes are needed for the Rust compiler to verify the compliance with the borrow rules, in particular for memory safety.
    The example from \autoref{lst:match} is about the lifetimes of borrows.
    The function \fun accepts a \Foo reference and if \codr{opt} has a value it should be printed and another function should be called.
    This leads to a compile time error in the current version of Rust.
    \autocite[2094-nll]{rust-rfc}

    \LstTikzFig{data/match.rs}{Self in Matches}{lst:match}

  \subsection{Borrowing from \texttt{RefCell<T>} Mutable in an Argumentlist}
    In \autoref{lst:argBorrow} an unfortunate side effect can be seen from using \RefCellT to enforce the borrow rules at runtime.
    \F is a \struct that is wrapped by \RefCellT and has the function \fun that needs a mutable \Self.
    In line 15 the program will terminate with a panic because the borrow rules where violated.
    The issue is not solved by introducing a scope around the first borrow and use of \codr{f}, like in \hun.
    Only the introduction of a scope local variable solves the issue, as seen in \nun.

    \LstTikzBox{\borrowArgOne}[.44\linewidth]{data/borrow_arg_1.rs}
    \LstTikzBox{\borrowArgTwo}[.44\linewidth]{data/borrow_arg_2.rs}[firstnumber=24]
    \begin{figure}[ht]
      \captionsetup{type=lstlisting}
      \centering
      \usebox{\borrowArgOne}\hfill%
      \usebox{\borrowArgTwo}\\\vspace*{.75em}%
      \caption{Borrowing from \RefCellT in argumentlist can lead to panics}\label{lst:argBorrow}
    \end{figure}

\end{document}
