%!TEX root = thesis.tex
% Copyright 2018 Roland Jäger
%
% Permission is hereby granted, free of charge, to any person obtaining a copy
% of this software and associated documentation files (the "Software"), to deal
% in the Software without restriction, including without limitation the rights
% to use, copy, modify, merge, publish, distribute, sublicense, and/or sell
% copies of the Software, and to permit persons to whom the Software is
% furnished to do so, subject to the following conditions:
%
% The above copyright notice and this permission notice shall be included in all
% copies or substantial portions of the Software.
%
% THE SOFTWARE IS PROVIDED "AS IS", WITHOUT WARRANTY OF ANY KIND, EXPRESS OR
% IMPLIED, INCLUDING BUT NOT LIMITED TO THE WARRANTIES OF MERCHANTABILITY,
% FITNESS FOR A PARTICULAR PURPOSE AND NONINFRINGEMENT. IN NO EVENT SHALL THE
% AUTHORS OR COPYRIGHT HOLDERS BE LIABLE FOR ANY CLAIM, DAMAGES OR OTHER
% LIABILITY, WHETHER IN AN ACTION OF CONTRACT, TORT OR OTHERWISE, ARISING FROM,
% OUT OF OR IN CONNECTION WITH THE SOFTWARE OR THE USE OR OTHER DEALINGS IN THE
% SOFTWARE.

\documentclass[
  a4paper,  % Standard format - only KOMAScript uses paper=a4 - https://tex.stackexchange.com/a/61044/9075
  twoside,  % we are optimizing for both screen and two-side printing. So the page numbers will jump, but the content is configured to stay in the middle (by using the geometry package)
  bibliography=totoc,
  headsepline,
  cleardoublepage=empty,
  parskip=half,
]{scrbook}
%%%%%%%%%%%%%%%%%%%%%%%%%%%%%%%%%%%%%%%%%%
% PACKAGES
%%%%%%%%%%%%%%%%%%%%%%%%%%%%%%%%%%%%%%%%%%
%---------------------------------
% Special ordering
%---------------------------------
\usepackage[noTeX]{mmap}% Better text copy support
\usepackage[scale=.9,ttdefault=true]{sourcecodepro}

%---------------------------------
% Language
%---------------------------------
\usepackage[T1]{fontenc} % utf8 <- produce real utf8 characters
\usepackage[utf8]{inputenc} % utf8 <- accept utf8 input characters
\usepackage{datetime}
\usepackage[english]{babel}
\usepackage{csquotes} % better quotes
\usepackage[
  backend=biber,
  sortcites=true,
  url=false,
  doi=true,
  eprint=false
]{biblatex}

%!TEX root = thesis.tex

%---------------------------------
% Meta Variables
%---------------------------------
\newcommand{\MetaUniversity}{University Stuttgart}
\newcommand{\MetaTitle}{Typesafe Parallel Serialization in SKilL/Rust}
\newcommand{\MetaAuthor}{Roland Jäger}
\newcommand{\MetaSemester}{SoSe 18}


%---------------------------------
% Utility
%---------------------------------
\usepackage[colorlinks]{hyperref} % hyper ref "addon"
\usepackage[all]{hypcap} % better linking/ref to floats
\usepackage{scrhack} % patches outdated float package
\usepackage{float} % better figure placement
\usepackage[justification=centering,format=plain]{caption} % Caption setup
\usepackage{amsmath} % Math stuff
\usepackage{amsfonts} % Math stuff
\usepackage{xspace} % for automagical spaces after commends
\usepackage[math]{blindtext}
\usepackage[automake,acronym,indexonlyfirst,nomain]{glossaries} % glossary and accronyms
\usepackage{pgffor} % used to iterate over a list
\usepackage{appendix}
\usepackage{upquote}
\usepackage{subfig}

%---------------------------------
% Style
%---------------------------------
\usepackage[automark]{scrlayer-scrpage}
\usepackage[
  left=3cm,right=3cm,top=2.5cm,bottom=2.5cm,
  headsep=18pt,
  footskip=30pt,
  includehead,
  includefoot
]{geometry}
\usepackage[
  activate={%
      true,%
      nocompatibility%
    }, % activate protrusion and expansion
  final, % enable microtype; use "draft" to disable
  tracking=true,%
  kerning=true,%
  spacing=true,%
  factor=1100, % add 10% to the protrusion amount (default is 1000)
  stretch=10, % reduce stretchability (default is 20/20)
  shrink=10 % reduce shrinkability (default is 20/20)
]{microtype} % Subliminal refinements towards typographical perfection - eg. Hypernation
\usepackage{lastpage} % For page of pages
\usepackage[svgnames,hyperref]{xcolor} % svg colors
\usepackage[
  language=english,
  title={\MetaTitle{}},
  author={\MetaAuthor{}},
  type=master,
  institute=iste,
  course={Softwareengineering},
  examiner={Prof.\ Dr.\ Erhard Plödereder},
  supervisor={Dr.\ Timm Felden},
  startdate={April 11, 2018},
  enddate={Octobre 11, 2018},
]{scientific-thesis-cover}

%---------------------------------
% Environment
%---------------------------------
\usepackage{longtable} % base for tabu (longtabu)
\usepackage{tabu} % Good table environment - best?
\usepackage{enumitem} % Improved enumerate/itemize/description env
\usepackage{graphicx} % for graphics -> \includegraphics
\usepackage{listings-rust}
\usepackage{listings}
\usepackage{tikz}
\usepackage{pgfplots}


%%%%%%%%%%%%%%%%%%%%%%%%%%%%%%%%%%%%%%%%%%
% CONFIGURATIONS
%%%%%%%%%%%%%%%%%%%%%%%%%%%%%%%%%%%%%%%%%%
%---------------------------------
% Image folders
%---------------------------------
\graphicspath{{./img}}

%---------------------------------
% Biographie
%---------------------------------
\addbibresource{lit.bib}

%---------------------------------
% Import the glossary
%---------------------------------
%!TEX root = thesis.tex

\makeglossaries

% Glossary - stuff the reader should know
% 1. \gls{ha} -> "HAZOP"
% 2. \gls{ha} -> "HAZOP"
\newglossaryentry{api}{
  name=API,
  description={Application Programming Interface},
  plural=APIs, % Optional
  descriptionplural={Application Programming Interfaces} % Optional
}
\newglossaryentry{oop}{
  name=OOP,
  description={Object Oriented Programing},
}
\newglossaryentry{css}{
  name=CSS,
  description={Cascading Style Sheet},
  plural=CSS,
  descriptionplural={Cascading Style Sheets}
}
\newglossaryentry{json}{
  name=JSON,
  description={JavaScript Object Notation},
}
\newglossaryentry{gui}{
  name=GUI,
  description={Graphic User Interface},
  plural=GUIs,
  descriptionplural={Graphic User Interfaces}
}

% Acronym - stuff the read probably doesn't know
% 1. \ac{ito} -> "Internet of Things (IoT)"
% 2. \ac{ito} -> "IoT"
\newglossaryentry{skill}{
  name=SKilL,
  description={Serialization Killer Language},
  first={\glsentrydesc{skill} (\glsentrytext{skill})},
}
\newglossaryentry{ir}{
  name=IR,
  description={Intermediate Representation},
  first={\glsentrydesc{ir} (\glsentrytext{ir})},
}
\newglossaryentry{ecs}{
  name=ECS,
  description={Entity Component System},
  first={\glsentrydesc{ecs} (\glsentrytext{ecs})},
  plural=ECSs, % optional
  descriptionplural={\glsentrydesc{ecs}s}, % optional
  firstplural={\glsentrydescplural{ecs} (\glsentryplural{ecs})} % optional
}



%---------------------------------
% Intex date format
%---------------------------------
\newdateformat{myInTextDateFormat}{%
  \ordinaldate{\THEDAY}\ \monthname[\THEMONTH] \THEYEAR%
}

%---------------------------------
% Normal Font for header and footer
%---------------------------------
\setkomafont{pageheadfoot}{\normalfont\sffamily}
\setkomafont{pagenumber}{\normalfont}

%---------------------------------
% Less space between list items
%---------------------------------
\setlist{itemsep=.5em}
\setlist{parsep=.5em}

%---------------------------------
% Meta data and Link Colour
%---------------------------------
\newcommand{\myShade}{70}

\definecolor{mylinkcolor}{RGB}{113, 31, 155}
\definecolor{mycitecolor}{RGB}{255, 189, 76}
\definecolor{myurlcolor}{RGB}{62, 106, 171}

\hypersetup{
  pdfauthor   = {\MetaAuthor},
  pdftitle    = {\MetaTitle},
  pdfsubject  = {\MetaTitle},
  pdfkeywords = {\MetaUniversity, \MetaTitle},
  colorlinks  = true,
  linkcolor   = mylinkcolor!\myShade!black,
  citecolor   = mycitecolor!\myShade!black,
  urlcolor    = myurlcolor!\myShade!black,
}

%---------------------------------
% Microtype optimisations
%---------------------------------
% http://www.khirevich.com/latex/microtype/
\SetProtrusion{encoding={*},family={bch},series={*},size={6,7}}
{1={ ,750},2={ ,500},3={ ,500},4={ ,500},5={ ,500},%
  6={ ,500},7={ ,600},8={ ,500},9={ ,500},0={ ,500}}
\SetExtraKerning[unit=space]
{encoding={*}, family={bch}, series={*}, size={footnotesize,small,normalsize}}
{%
\textendash={400,400}, % en-dash, add more space around it
"28={ ,150}, % left bracket, add space from right
"29={150, }, % right bracket, add space from left
\textquotedblleft={ ,150}, % left quotation mark, space from right
\textquotedblright={150, } % right quotation mark, space from left
}
\SetExtraKerning[unit=space]
{encoding={*}, family={qhv}, series={b}, size={large,Large}}
{%
  1={-200,-200},%
  \textendash={400,400}%
}
\SetTracking{encoding={*}, shape=sc}{20}
\microtypecontext{spacing=nonfrench}

%---------------------------------
% Caption format
%---------------------------------
\captionsetup{
  format=hang,
  labelfont=bf,
  justification=justified,
  %single line captions should be centered, multiline captions justified
  singlelinecheck=true
}

%-----------------------------------------
% Rust language
%-----------------------------------------
\makeatletter
% here is a macro expanding to the name of the language
% (handy if you decide to change it further down the road)
\newcommand\language@rust{myRust}

\expandafter\expandafter\expandafter\lstdefinelanguage
\expandafter{\language@rust}
{
  language=Rust,
  alsoletter={!'},
  morekeywords=[7]{'a, 'b, 'c, 'd, 'e, 'f, 'g, 'h, 'i, 'j, 'k, 'l, 'm, 'n, 'o, 'p, 'q, 'r, 's, 't, 'u, 'v, 'w, 'x, 'y, 'z, 'A, 'B, 'C, 'D, 'E, 'F, 'G, 'H, 'I, 'J, 'K, 'L, 'M, 'N, 'O, 'P, 'Q, 'R, 'S, 'T, 'U, 'V, 'W, 'X, 'Y, 'Z},
}
\makeatother

%---------------------------------
% Listings style
%---------------------------------
% This is a hack for sourcecode pro
% See https://github.com/silkeh/latex-sourcecodepro/issues/4
\makeatletter
\lst@CCPutMacro
\lst@ProcessOther {"60}{%
\lst@ttfamily{\raisebox{4pt}{\`{}}}% used with ttfamily
\textasciigrave%
}% used with other fonts
\@empty\z@\@empty
\makeatother
\lstdefinestyle{myStyle}{
belowcaptionskip=1\baselineskip,
breaklines=true,
xleftmargin=\parindent,
basicstyle={\color{black}\small\ttfamily},
keywordstyle=[1]{\bfseries\color[rgb]{0,.3,.7}},
keywordstyle=[2]{\bfseries\color[rgb]{0,.1,.3}},
keywordstyle=[3]{\bfseries\color{Orange}},
keywordstyle=[4]{\bfseries\color{Coral}},
keywordstyle=[5]{\bfseries\color[rgb]{0,.2,.5}},
keywordstyle=[6]{\bfseries\color{Teal}},
keywordstyle=[7]{\bfseries\color{Brown}},
commentstyle={\color{DimGray}},
stringstyle={\color{DarkGreen}},
numberstyle={\tiny\color{Gray}},
identifierstyle={\color{black}},
breaklines=true,
breakatwhitespace=true,
keepspaces,
aboveskip=0.8em,
belowcaptionskip=-1em,
literate={%
{Ö}{{\"O}}1%
{Ä}{{\"A}}1%
{Ü}{{\"U}}1%
{ß}{{\ss}}1%
{ü}{{\"u}}1%
{ä}{{\"a}}1%
{ö}{{\"o}}1%
{~}{{\textasciitilde}}1%
{\.}{}{0\discretionary{.}{}{.}}},
}

\lstdefinestyle{myEnvStyle}{
  style=myStyle,
  frameshape={RYR}{Y}{Y}{RYR},
  captionpos=b,
  numbers=left,
}

%---------------------------------
% Listing shorthandles
%---------------------------------
\newcommand\cod{\lstinline[style=myStyle]} % the body {} is evaluated by lstinline - this leads to the highest compatibility.
\newcommand\codc{\lstinline[language=C++, style=myStyle]} % the body {} is evaluated by lstinline - this leads to the highest compatibility.
\newcommand\codr{\lstinline[language=myRust, style=myStyle]} % the body {} is evaluated by lstinline - this leads to the highest compatibility.
\newcommand\codj{\lstinline[language=Java, style=myStyle]} % the body {} is evaluated by lstinline - this leads to the highest compatibility.

%---------------------------------
% Setup Listings cations for subfigures
%---------------------------------
\def\sublstlistingautorefname{\lstlistingname}
\AfterPreamble{\newsubfloat{lstlisting}}

%---------------------------------
% Prevent single page figures with no text
%---------------------------------
\renewcommand{\topfraction}{0.85}
\renewcommand{\bottomfraction}{0.95}
\renewcommand{\textfraction}{0.1}
\renewcommand{\floatpagefraction}{0.75}

%---------------------------------
% No backwardscompatibility
%---------------------------------
\pgfplotsset{compat=1.16}

%---------------------------------
% Fix hypernation
%---------------------------------
\hyphenation{Zei-len-um-bruch}

%%%%%%%%%%%%%%%%%%%%%%%%%%%%%%%%%%%%%%%%%%
% COMMANDS
%%%%%%%%%%%%%%%%%%%%%%%%%%%%%%%%%%%%%%%%%%
%---------------------------------
% Ref footnotes
%---------------------------------
\makeatletter
\newcommand{\myFootnoteref}[1]{\protected@xdef\@thefnmark{\ref{#1}}\@footnotemark}
\makeatother

%---------------------------------
% hyphen used between two capitals
%---------------------------------
\newcommand{\myCapitalHyphen}{%
  \raisebox{0.24ex}{%
    \resizebox{0.4em}{%
      \height%
    }{-}%
  }%
  \kern-0.07em%
}

%---------------------------------
% href Link as foot note
%---------------------------------
\newcommand{\myFnUrl}[2]{%
  \ifthenelse{\equal{#1}{}}{%
    \footnote{\url{#2}}%
  }{%
    \href{#2}{#1}\footnote{ \url{#2}}%
  }%
}

%---------------------------------
% Horizontal line for title page
%---------------------------------
\newcommand{\HRule}{\rule{\linewidth}{0.2mm}}

%-------------------
% Command to break text without spaces or hyphens
%-------------------
\newcommand\myTextBreaker[1]{\tbhelp#1\relax\relax\relax}
\def\tbhelp#1#2\relax{{#1}\penalty0\ifx\relax#2\else\tbhelp#2\relax\fi}

%-------------------
% Command to break urls without spaces or hyphens
%-------------------
\makeatletter
\catcode`\%=12
\newcommand\pcnt{\%}
\catcode`\%=14

\catcode`\_=12
\newcommand\unsc{\_}
\catcode`\_=8

\newcommand*\myURLBreaker{\myURLBreaker@iii\myURLBreaker@i}
\newcommand*\myURLBreaker@iii[1]{\begingroup\catcode`\_12\catcode`\%12 #1}
\newcommand*\myURLBreaker@i[1]{\href{#1}{\ubhelp#1\relax\relax\relax}\endgroup}
\def\ubhelp#1#2\relax{{#1}\penalty0\ifx\relax#2\else\ubhelp#2\relax\fi}
\makeatother

%-------------------
% Convenience command to create a nested table
%-------------------
\newcommand{\specialcell}[2][l]{%
  \setlength{\extrarowheight}{0pt}%
  \def\arraystretch{1}%
  \tabulinesep=.025em%
  \begin{tabular}[t]{@{}#1@{}}#2\end{tabular}%
}

%%%%%%%%%%%%%%%%%%%%%%%%%%%%%%%%%%%%%%%%%%
% Environments
%%%%%%%%%%%%%%%%%%%%%%%%%%%%%%%%%%%%%%%%%%

%%%%%%%%%%%%%%%%%%%%%%%%%%%%%%%%%%%%%%%%%%
% Special treatment
%%%%%%%%%%%%%%%%%%%%%%%%%%%%%%%%%%%%%%%%%%
\usepackage{subfiles}
