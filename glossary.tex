%!TEX root = thesis.tex

\makeglossaries

% Glossary - stuff the reader should know
% 1. \gls{ha} -> "HAZOP"
% 2. \gls{ha} -> "HAZOP"
\newglossaryentry{api}{
  name=API,
  description={Programmierschnittstell, von application programming interface},
  plural=APIs, % Optional
  descriptionplural={Programmierschnittstellen, von application programming interfaces} % Optional
}
\newglossaryentry{oop}{
  name=OOP,
  description={Object Oriented Programing},
}
\newglossaryentry{css}{
  name=CSS,
  description={Cascading Style Sheet},
  plural=CSS,
  descriptionplural={Cascading Style Sheets}
}
\newglossaryentry{gui}{
  name=GUI,
  description={Graphic User Interface},
  plural=GUIs,
  descriptionplural={Graphic User Interfaces}
}

% Acronym - stuff the read probably doesn't know
% 1. \ac{ito} -> "Internet of Things (IoT)"
% 2. \ac{ito} -> "IoT"
\newglossaryentry{fm}{
  name=FM,
  description={Fehlerzustand},
  first={\glsentrydesc{fm} (\glsentrytext{fm}, von Failure Mode)},
  plural=FMs, % optional
  descriptionplural={Fehlerzustände}, % optional
  firstplural={\glsentrydescplural{fm} (\glsentryplural{fm}, von Failure Modes)} % optional
}
\newglossaryentry{skill}{
  name=SKilL,
  description={Serialization Killer Language},
  first={\glsentrydesc{skill} (\glsentrytext{skill})},
}
\newglossaryentry{ir}{
  name=IR,
  description={Intermediate Representation},
  first={\glsentrydesc{ir} (\glsentrytext{ir})},
}
\newglossaryentry{ecs}{
  name=ECS,
  description={Entity Component System},
  first={\glsentrydesc{ecs} (\glsentrytext{ecs})},
  plural=ECSs, % optional
  descriptionplural={\glsentrydesc{ecs}s}, % optional
  firstplural={\glsentrydescplural{ecs} (\glsentryplural{ecs})} % optional
}

